Os artigos relacionados a este trabalho estão ligados a métodos e satisfação do ensino em geral e especializado na Engenharia de \textit{Software}, desenvolvimento ágil e na metodologia \textit{Lean}. Como este trabalho busca analisar a aplicação de um modelo \textit{Lean} para o ensino, são citados inicialmente artigos que abordam como ensinar de maneira efetiva e satisfatória.

\nocite{EnsinoEng2014}Santos et al. (2014) abordam em seu artigo métodos e experimentos de ensino na Engenharia de \textit{Software} como disciplina da área de informática. O artigo sempre aborda o fato do amplo conteúdo que a Engenharia de \textit{Software} abrange e sua importância para a qualidade dos sistemas gerados. Devido a grande quantidade de conteúdo, se tornou necessário a utilização de métodos diferentes de ensino na tentativa de repassar a base da Engenharia de \textit{Software} de forma adequada. Para atingir o objetivo do artigo, a principal abordagem utilizada pelos autores, foi a aplicação de atividades práticas na disciplina, método que se assemelha a um dos princípios do \textit{Lean Learning}.

\nocite{chatley2017lean}Chatley e Field (2017) abordam como se aprende algo, reforçando a ideia de práticas durante o aprendizado, em vez de apenas materiais de pesquisa e textos repetitivos dos livros escolares. Segundo a metodologia \textit{Lean}, é possível evitar o desperdício de recursos materiais e de tempo caso o ensino seja conduzido da maneira correta. Como eles utilizaram o \textit{Lean} no ensino, prezaram fortemente por um \textit{feedback} rápido e ciclos iterativos curtos, para assim manter o contato com os 150 alunos da Universidade Imperial. Decidiram utilizar essa abordagem para familiarizar os alunos com o mercado atual. Também defendem que o ensinamento real vem da prática, sendo o ensino teórico apenas a base. A abordagem é positiva, visto que incentivam o uso de técnicas do mercado que se demonstraram bem sucedidas, como a programação em pares e ideias semelhantes a \textit{Hackathons}. Como este trabalho aplica uma análise sobre as práticas realizadas em disciplinas em conjunto com o \textit{Lean Learning}, o artigo de Chatley e Field é uma referência fundamental.

Dentre os artigos que abordam as metodologias ágeis e \textit{Lean},\nocite{poth2019lean} Poth et al. (2019) apresentam a eficiência e a melhora produtiva ao se utilizar o \textit{Lean} e metodologias ágeis em conjunto. Seu artigo busca analisar as abordagens \textit{Lean} e \textit{Agile SPI (Software Process Improvement)} em ambientes de desenvolvimento tradicionais e ágeis e demonstrar que em ambos os cenários é possível buscar essas abordagens. São pontuados os princípios chave do \textit{Lean Software Development} e então revisitam o Manifesto Ágil. Essa construção permite ao leitor perceber as conexões existentes desde o surgimento do \textit{Lean} e seus processos de qualidade e melhoria e seus princípios. Então, há uma análise de SPI em ambientes tradicionais e ágeis e são discutidas as abordagens para cada um. Como este trabalho visa analisar as consequências da aplicação dessas metodologias em uma graduação em Engenharia de \textit{Software}, o trabalho de Poth et al. (2019) serve como referência a esta pesquisa.

Devido ao tema de \textit{Lean Learning} aplicado à Engenharia de \textit{Software} não ter sido muito explorado ainda, foram utilizados artigos que abordam o \textit{Lean Learning} aplicado em outras áreas. Alguns exemplos desses trabalhos são a pesquisa de Marques et al. (2017)\nocite{marques2017enhancing}, de Yadav et al. (2018)\nocite{Yadav2018} e o livro de Parsons and MacCallum (2019) \nocite{parsons2019agile}. Como esses artigos não estão relacionados à Engenharia de \textit{Software}, servirão como auxílio na compreensão da abordagem \textit{Lean} aplicada ao ensino.