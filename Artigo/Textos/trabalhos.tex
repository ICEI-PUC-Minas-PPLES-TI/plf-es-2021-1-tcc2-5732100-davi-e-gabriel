Os artigos relacionados a este trabalho estão ligados a métodos e satisfação do ensino em geral e especializado na engenharia de \textit{software}, desenvolvimento ágil e na metodologia \textit{Lean}. Como este trabalho busca analisar a aplicação de um modelo \textit{Lean} para o ensino, são citados inicialmente artigos que abordam como ensinar de maneira efetiva e satisfatória.

%No artigo de Nowostawski et al. (2018)\nocite{gamifying2018}, é abordado um método de ensino que tem crescido ultimamente, a  \textit{Gamification}. Para este trabalho, há um maior proveito da sessão que aborda a satistação dos aprendizes. O artigo quer demonstrar como a gamificação pode apresentar melhoras no ensino da engenharia de \textit{software}, tanto em qualidade quanto em interesse dos alunos. Para isso, eles criaram uma metodologia de ensino à parte, chamada \textit{The Game of Reading and Discussion}. Durante 4 anos (2015-2018) essa metodologia foi aplicada e consistia basicamente em fazer os alunos lerem artigos, elaborarem perguntas sobre os artigos e após isso, realizar uma reflexão dos artigos lidos e avaliar a reflexão de algum outro aluno. E ao final os números mostram que os alunos estão mais engajados nas aulas.

%\nocite{marques2017enhancing}Marques et al. (2017) citam sobre os métodos ágeis no ensino do desenvolvimento de \textit{software}. Grande parte das instituições de ensino começam a lecionar o desenvolvimento de \textit{software} de maneira extremamente focada no desenvolvimento ágil. Porém nem sempre essa pode ser a maneira mais adequada de se ensinar a construir \textit{softwares} para programadores novatos. O método de reflexão semanal de desempenho (RWM, ou Reflexive Weekly Monitoring) é abordado no artigo, alegando que este método de início seria mais efetivo e coordenado, além de os alunos se sentirem mais satisfeitos e prestativos com a sua equipe. O RWM busca principalmente desenvolver as competências dos alunos, como a vontade de aprender, adaptação a ferramentas entre outras. Como o artigo trata de uma conexão entre metodologias ágeis -- que são técnicas utilizadas pelo mercado -- e aprendizagem, seu tema conecta-se ao deste trabalho, na medida em que analisa a combinação entre métodos ágeis e uma metodologia para avaliação de desempenho na aprendizagem.
\nocite{EnsinoEng2014}Santos et al. (2014) abordam em seu artigo métodos e experimentos de ensino na engenharia de \textit{software} como disciplina da área de informática. O artigo sempre aborda o fato do amplo conteúdo que a engenharia de \textit{software} abrange e sua importância para a qualidade dos sistemas gerados. Devido a grande quantidade de conteúdo, se tornou necessário a utilização de métodos diferentes de ensino na tentativa de repassar a base da engenharia de \textit{software} de forma adequada. Para atingir o objetivo do artigo, a principal abordagem utilizada pelos autores, foi a aplicação de atividades práticas na disciplina, método que se assemelha a um dos princípios do \textit{Lean Learning}.

\nocite{chatley2017lean}Chatley e Field (2017) abordam como aprende-se algo, reforçando a ideia de práticas durante o aprendizado, em vez de apenas materiais de pesquisa e textos repetitivos dos livros escolares. Segundo a metodologia \textit{Lean}, é possível evitar o desperdício de recursos materiais e de tempo caso o ensino seja conduzido da maneira correta. Como eles utilizaram o \textit{Lean} no ensino, prezaram fortemente por um \textit{feedback} rápido e ciclos iterativos curtos, para assim manter o contato com os 150 alunos da Universidade Imperial. Decidiram utilizar essa abordagem ágil para familiarizar os alunos com o mercado atual. Também defendem que o ensinamento real vem da prática, sendo o ensino teórico apenas a base. A abordagem é positiva, visto que incentivam o uso de técnicas do mercado que se demonstraram bem sucedidas, como a programação em pares e ideias semelhantes a \textit{Hackathons}. Como este trabalho aplica uma análise sobre as práticas realizadas em disciplinas em conjunto com o \textit{Lean Learning}, o artigo de Chatley e Field é uma referência fundamental.

%Metodologias ágeis
Dentre os artigos que abordam as metodologias ágeis e \textit{Lean},\nocite{poth2019lean} Poth et al. (2019) apresentam a eficiência e a melhora produtiva ao se utilizar o \textit{Lean} e metodologias ágeis em conjunto. Seu artigo busca analisar as abordagens \textit{Lean} e \textit{Agile SPI (Software Process Improvement)} em ambientes de desenvolvimento tradicionais e ágeis e demonstrar que em ambos os cenários é possível buscar essas abordagens. São pontuados os princípios chave do \textit{Lean Software Development} e então revisitam o Manifesto Ágil. Essa construção permite ao leitor perceber as conexões existentes desde o surgimento do \textit{Lean} e seus processos de qualidade e melhoria e seus princípios. Então, há uma análise de SPI em ambientes tradicionais e ágeis e são discutidas as abordagens para cada um. Como este trabalho visa analisar as consequências da aplicação dessas metodologias em uma graduação em Engenharia de \textit{Software}, o trabalho de Poth et al. (2019) serve como referência a esta pesquisa.

%\nocite{fernandes2019identifying}Fernandes e Barcelos (2019), em sua pesquisa sobre a aplicação de metodologias ágeis a projetos de reengenharia de \textit{software}, fazem menção aos resultados observados por outros trabalhos. Estes, mostram que os métodos ágeis mais utilizados na indústria brasileira de \textit{software} são \textit{Scrum} e XP, e os benefícios de seu uso. Os resultados do projeto foram comparados com a percepção do time de desenvolvimento por meio de entrevistas semiestruturadas, com a análise dos artefatos do projeto e as melhores práticas propostas na literatura para entender se os resultados mencionados nesta última seriam confirmados pela prática. Os resultados da pesquisa de Fernandes e Barcelos indicam que o produto da reengenharia foi bem sucedido, de acordo com os \textit{stakeholders} do projeto. A utilização de metodologias ágeis foi reconhecida como fator de sucesso.

%O trabalho de Miranda et al. (2019)\nocite{wei2019}, que estuda o impacto do \textit{Scrum} no sucesso de projetos de \textit{software}, sinaliza sobre a aplicação de metodologias ágeis no meio acadêmico. Nas considerações finais do artigo, destaca-se o engajamento dos alunos que participaram do estudo, em relação ao \textit{Scrum} e seus benefícios. Apesar de exigir maior dedicação para se aprender a metodologia nova, o resultado observado foi positivo.

%A decisão de criar questionários e realizar entrevistas para obter os resultados do trabalho partiu de dois trabalhos, sendo eles o de Bandeira (2003)\nocite{Questionarios1} e de Henkel (2017)\nocite{Questionarios2}. Em ambos são apresentados métodos para análise de respostas abertas e fechadas em questionários. Para os resultados deste trabalho, utiliza-se parte dos métodos descritos por ambos, onde é necessário interpretar a fundo as respostas abertas e explorar as métricas obtidas com as respostas fechadas.