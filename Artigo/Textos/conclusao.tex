Foi observado pelos pesquisadores uma adesão maior dos alunos nas práticas 1 e 3, onde a comunicação era um dos temas principais das práticas. A prática 2, por se tratar de um exercício de lógica, aparentou entediar os alunos mais experientes do curso, e assustar alunos mais novos que não estariam tão preparados para a mesma. Acredita-se que com uma adequação melhor da prática, e um estímulo maior à competição, como recompensar mais pontos ou até mesmo pagar um curso ao grupo vencedor poderia resultar em uma adequação e participação melhor da prática 2.

As práticas 1 e 3 se mostraram adequadas ao contexto, e por isso também foram satisfatórias aos alunos e aplicadores, inclusive durante a aplicação dessas práticas foi observado uma interação mais fluida entre ambas as partes. Principalmente na prática 3, abordando um tema relacionado a pesquisa aqui proposta, o interesse de vários alunos acerca dos resultados apresentados na Seção 5 foi despertado.

Como conclusão, tem-se que o ensino orientado a aulas práticas apresentou resultados satisfatórios, vários alunos demonstraram grande interesse na pesquisa, nas atividades apresentadas e na possível melhoria do sistema de ensino e do curso em geral. Também comentaram que conseguiram perceber bem a integração com o mercado nas atividades propostas e podem ser uma boa base introdutória de alunos mais novos no curso ao mercado.