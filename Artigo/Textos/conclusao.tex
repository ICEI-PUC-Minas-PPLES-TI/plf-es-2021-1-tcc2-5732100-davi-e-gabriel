Como conclusão, tem-se que o ensino orientado a aulas práticas apresentou resultados satisfatórios. Vários alunos demonstraram grande interesse na pesquisa, nas atividades apresentadas e na possível melhoria do ensino orientado pelos princípios \textit{Lean}. Também comentaram que conseguiram perceber a relação das atividades propostas com o mercado de trabalho de \textit{software} e elas podem servir, em certo nível, como base introdutória para técnicas e práticas utilizadas no mercado.

De acordo com a análise de resultados, é possível concluir, em relação aos objetivos deste trabalho, que: i) alunos e professores ficaram satisfeitos com a abordagem -- mesmo que reduzida à etapa Medir -- que foi feita do \textit{Lean Learning}; ii) a proximidade das práticas ao mercado de trabalho de \textit{software} atendeu às expectativas. Entretanto, a adequação das práticas poderia ter sido melhor trabalhada. Houveram queixas em relação ao tempo disponibilizado e divergência em relação a complexidade das práticas, visto que as mesmas foram aplicadas em diversos períodos do curso. A metodologia parece aplicável e viável no contexto das disciplinas estudadas.

Também é possível concluir que um dos caminhos para, pelo menos, iniciar a implementação da abordagem do \textit{Lean Learning} poderia se dar a partir da etapa Medir. Os professores que decidissem adotá-la, poderiam começar utilizando suas próprias práticas, aplicando-as e medindo-as por meio de questionários ou outras ferramentas de coleta de dados, de forma a buscar compreender como elas podem ser continuamente melhoradas, a partir de \textit{feedbacks} constantes dos alunos. Com esses dados em mãos, os professores poderiam implementar, aos poucos e de maneira cíclica, as demais etapas ilustradas na Figura 1, até o processo começar a rodar naturalmente. 

Uma limitação que requer atenção ao avaliar a viabilidade da implementação da abordagem em uma disciplina é o estudo da proximidade existente entre as duas -- neste trabalho, foram selecionadas para estudo somente disciplinas que tinham alguma relação com \textit{Lean Management} ou com áreas mais técnicas do mercado de trabalho de \textit{software}. Também é possível que existam disciplinas que não se encaixem com o modelo proposto, visto que o espaço amostral desta pesquisa se limitou a apenas cinco disciplinas.

Este trabalho trouxe duas contribuições: uma prática e uma teórica. A prática se exprime nos resultados e conclusões apresentadas sobre a eficácia do uso de práticas no ensino. A teórica se dá pelo modelo implementado pela pesquisa, que pode servir como ferramenta para outros estudos. Trabalhos futuros podem vir a investigar a viabilidade da implementação do \textit{Lean Learning} ao curso de Engenharia de \textit{Software} como um todo, analisando outros critérios e abordagens para disciplinas além dos que foram analisados neste estudo. Também pode-se vir a investigar a aplicação do \textit{Lean Learning} a outros cursos de tecnologia e áreas afins da Engenharia de \textit{Software}, uma vez que é possível que o problema estudado neste trabalho também exista para demais cursos.