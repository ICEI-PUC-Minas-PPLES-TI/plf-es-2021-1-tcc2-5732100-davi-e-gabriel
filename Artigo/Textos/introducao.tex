A área de desenvolvimento de \textit{software} atualmente conta, em grande parte, com metodologias de desenvolvimento ágil e práticas derivadas dos princípios do \textit{Lean Management} \cite{poth2019lean}. Oriundo do Japão, o \textit{Lean} consiste numa filosofia de gestão criada a partir do modelo toyotista de produção, que busca eliminar ao máximo os desperdícios dos recursos envolvidos na cadeia produtiva, resolver problemas sistematicamente e mudar a forma de visualizar e gerenciar um negócio \cite{shingo2019study}. Por se tratar de um modelo orientado a processos, métricas e conceitos, o \textit{Lean} é amplamente aplicável a empresas de diversos setores.

Estudos recentes apontam a dificuldade em lecionar com qualidade o conteúdo relacionado à Engenharia de \textit{Software}, devido a quantidade de informações e relevância da área \cite{EnsinoEng2014} \cite{chatley2017lean} \cite{marques2017enhancing}. Com o objetivo de transmitir o máximo de conteúdo da área, costuma-se realizar atividades práticas com os alunos \cite{EnsinoEng2014}. O livro de Chatley, ``Agile and Lean Concepts for Teaching and Learning" \nocite{ChatleyBook2019}, traz maneiras de se aplicar o \textit{Lean Learning} no ensino da Engenharia de \textit{Software}. Também mostra dois caminhos a serem seguidos quando se utiliza o \textit{Lean Learning}. O caminho tradicional visa modificar o ensino comum buscando \textit{feedback} constante e o caminho orientado a projetos busca ensinar por meio da prática simulada de determinada teoria.

O principal problema que este trabalho analisa é a ausência de medidas da eficácia de ferramentas \textit{Lean} no ensino da Engenharia de \textit{Software}. Para tanto, o estudo avalia a aplicação de práticas em sala de aula e explora aspectos do nível de satisfação, adequação e proximidade com o mercado de trabalho, buscando assim aproximar o aprendizado aos conceitos ágeis do \textit{Lean}. Em função deste problema, este trabalho se propõe a investigar a implementação da etapa \textit{Measure} (Medir) do \textit{Lean Learning} e seu potencial de trazer melhoria contínua para o aprendizado da Engenharia de \textit{Software}. A escolha dessa etapa se dá pelo fato de ser nela que ocorre a coleta de informação, a fim de receber \textit{feedback} qualitativo dos alunos. É a partir desse retorno que se pode buscar melhorias no ensino, fazendo desta etapa a mais importante para o processo.

O objetivo geral deste trabalho é avaliar empiricamente a aplicação da fase de medição da metodologia \textit{Lean Learning} no ensino da Engenharia de \textit{Software}, a fim de potencializar a produtividade e qualidade do ensino. Também existem objetivos específicos, estes sendo: i) propor uma abordagem de aplicação da etapa \textit{Measure} do \textit{Lean Learning} em disciplinas de Engenharia de \textit{Software}; ii) avaliar a aplicabilidade da metodologia no contexto da Engenharia de \textit{Software}; iii) obter a percepção de professores e alunos acerca da abordagem proposta.

O trabalho se divide em seis seções. A Seção 2 apresenta mais conceitos importantes e detalhes sobre as abordagens de implementações da metodologia \textit{Lean Learning}. A Seção 3 descreve os principais trabalhos relacionados. A Seção 4 detalha e discute a metodologia aplicada no experimento. Na Seção 5 ocorre a análise estatística dos resultados obtidos. Na Seção 6 encontram-se a conclusão e as considerações finais deste trabalho.