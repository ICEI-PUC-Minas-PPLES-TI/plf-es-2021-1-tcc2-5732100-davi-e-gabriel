%Apresentando cenário atual do mercado de software, Definindo o Lean e Citando problemas do mercado atual e assim gerando o Lean Agile
A área de desenvolvimento de software atualmente conta, em grande parte, com metodologias de desenvolvimento ágil. Recentemente, outro modelo produtivo que tem crescido em todo o mundo é o \textit{Lean}. Oriundo do Japão, o \textit{Lean} consiste numa filosofia de gestão criada a partir do modelo toyotista de produção, que busca eliminar ao máximo os desperdícios dos recursos envolvidos na cadeia produtiva, resolver problemas sistematicamente e mudar a forma de visualizar e gerenciar um negócio \cite{poth2019lean}. Por se tratar de um modelo orientado a processos, métricas e conceitos, o \textit{Lean}é amplamente aplicável a empresas de diversos setores.

%A partir desse parágrafo fiz pequenas mudanças
É possível verificar em diversos artigos a dificuldade em lecionar com qualidade o conteúdo relacionado a engenharia de \textit{software}, devido a quantidade de informações e relevância da área. Com o objetivo de consolidar o máximo de informações da área, costuma-se realizar atividades práticas com os alunos \cite{EnsinoEng2014}. Mas ainda existem métodos com poucos estudos sobre eles, como é o caso do \textit{Lean Learning}.
Observando as tendências de mercado e diversos artigos que abordam o ensino da engenharia de \textit{Software}, há dificuldade em manter o ensino concreto de seus métodos e ferramentas em seu estado-da-arte -- e nisso o \textit{Lean Learning} pode auxiliar \cite{chatley2017lean}. O principal problema que este trabalho tenta resolver é a falta de conhecimento da aplicabilidade da etapa \textit{Measure} (Medir), do \textit{Lean Learning} ao ensino da Engenharia de \textit{Software}.

No livro de Chatley, ``Agile and Lean Concepts for Teaching and Learning" \nocite{ChatleyBook2019}, traz maneiras de se aplicar o \textit{Lean Learning} no ensino da engenharia de \textit{software}. Também mostra dois caminhos a serem seguidos quando se utiliza o \textit{Lean Learning}. O caminho ``tradicional", visa modificar o ensino comum buscando \textit{feedback} constante e o caminho ``orientado a projetos" que busca ensinar por meio da prática simulada de determinada teoria.
%a dificuldade existente em manter o ensino dos cursos de tecnologia o mais atual possível -- do ponto de vista de técnicas e ferramentas --, visto que a área de Tecnologia da Informação está em constante transformação.

%Durante a graduação em Engenharia de \textit{Software}, pôde-se observar diversos comentários de alunos em sala de aula, de que só aprenderam na prática as técnicas e ferramentas empregadas no mercado de trabalho depois que começaram a trabalhar na área, o que leva a questionamentos sobre como é possível agregar mais destes conhecimentos. Um dos possíveis caminhos seria a formação e o fortalecimento de alianças com o mercado de trabalho, buscando monitorá-lo e integrá-lo através de eventos e atividades, de modo a agregar mais conhecimentos sobre como é o trabalho de um engenheiro de \textit{software} moderno na prática e o quão diverso ele pode ser. %E esta é uma medida que vem sendo implementada pela coordenação e pelo corpo docente de Engenharia de \textit{Software} da PUC Minas.

Este trabalho se propõe a investigar a possibilidade da implementação da etapa \textit{Measure} (Medir) do \textit{Lean Learning} e seu potencial de trazer melhoria contínua para o aprendizado da Engenharia de \textit{Software}. A finalidade é auxiliar no acompanhamento contínuo dos avanços da área e das expectativas do mercado de trabalho e dos alunos. A escolha da etapa Medir do \textit{Lean Learning} se dá pelo fato de essa etapa ser onde ocorre a coleta de informação, a fim de receber \textit{feedback} quantitativo dos alunos. É a partir desse retorno que se pode buscar melhorias no ensino, fazendo desta etapa a mais importante para o processo.
%Levando em consideração o tempo de que se dispõe para a realização do Trabalho de Conclusão de Curso, não seria factível uma tentativa de experimentar o \textit{Lean Learning} completo; para fazê-lo, seriam necessários vários semestres de experimentação. Portanto, este trabalho se restringe à etapa \textit{Measure}, experimentada durante um semestre. %acompanhar de forma ágil as tendências do mercado e constantes mudanças da área.

%Objetivos
%Geral

%levando em consideração que é esperada uma melhor adequação ao mercado

O objetivo geral deste trabalho é avaliar empiricamente a aplicação da fase de medição da metodologia \textit{Lean Learning} no ensino da Engenharia de \textit{Software}, a fim de potencializar a produtividade e qualidade do ensino. Também existem objetivos específicos, estes sendo: i) propor uma abordagem de aplicação da etapa \textit{Measure} do \textit{Lean Learning} em disciplinas de Engenharia de \textit{Software}; ii) avaliar a aplicabilidade da metodologia no contexto da Engenharia de \textit{Software}; iii) obter a percepção de professores e alunos acerca da abordagem proposta."

%melhorar desempenho dos alunos de Engenharia de Software nas disciplinas lecionadas; desenvolver suas habilidades e capacidades para trabalho em equipe, como proatividade, produtividade, comunicação, autonomia e consciência individual e coletiva, fundamentais para trabalhar numa realidade

%Específicos
%Objetivos Especificos:
%Melhor desempenho dos alunos de engenharia de software nas matérias lecionadas
%Maior proatividade, produtividade e consciência dos alunos
%Profissionais melhorarem sua comunicação com futuros companheiros de trabalho

%Resultados esperados
Espera-se que, ao final da pesquisa, seja possível obter percepções de alunos e professores acerca do aprendizado, de modo a analisar possíveis novas abordagens de ensino para a Engenharia de \textit{Software}, visando agregar ainda mais na formação dos alunos. Também espera-se conseguir analisar a possibilidade de implementação da metodologia na Engenharia de \textit{Software} e propor uma possível abordagem para tal. 

O restante deste trabalho está dividido em quatro seções. A Seção 2 apresenta mais conceitos importantes e detalhes sobre as abordagens de implementações da metodologia \textit{Lean Learning}. A Seção 3 descreve os principais trabalhos relacionados. A Seção 4 detalha e discute a metodologia proposta.